\chapter{Desenvolvimento}
\label{chapter:desenvolvimento}

Este capítulo aborda os seguintes tópicos:
\begin{itemize}
\item Na Seção 3.1, descrevemos a metodologia utilizada.
\item Na Seção 3.2, os resultados são mostrados e discutidos.
\end{itemize}

\section{Métodos}

Foi planejado e realizado um estudo que visa avaliar estado manutenibilidade
do Linux Kernel, utilizando o sofware SonarQube para a medição. Os
requerimentos necessários pelo método SQALE foram as fornecidas pelo
plugin \textbf{sonar-cxx\cite{Sonarcxx2016}}. O software realiza
uma leitura completa do código fonte da aplicação a ser analisada,
identificando os trechos não adequados de acordo com os requerimentos
definidos. Para cada não conformidade encontrada, há uma função de
remediação definida com o tempo estimado de remediação cada ocorrência
encontrada. Foram consideradas 6 versões de lançamento estáveis foram
analisadas dentro das versões mais recentes, procurando avaliar o
estado atual da manutenibilidade do software. 

\section{Resultados e Discussão}

\begin{table}[h]
\begin{centering}
\begin{tabular}{|c|c|c|c|c|c|c|}
\hline 
{\scriptsize{}Característica/Versão} & {\scriptsize{}3.19} & {\scriptsize{}4.0} & {\scriptsize{}4.1} & {\scriptsize{}4.2} & {\scriptsize{}4.3} & {\scriptsize{}4.4}\tabularnewline
\hline 
\hline 
{\scriptsize{}Número de Linhas} & {\scriptsize{}17.396.102 } & {\scriptsize{}17.557.054 } & {\scriptsize{}17.724.734 } & {\scriptsize{}18.486.941 } & {\scriptsize{}18.766.571 } & {\scriptsize{}19.003.936 }\tabularnewline
\hline 
{\scriptsize{}Debito Técnico} & {\scriptsize{}311d} & {\scriptsize{}309d} & {\scriptsize{}330d} & {\scriptsize{}338d} & {\scriptsize{}342d} & {\scriptsize{}343d}\tabularnewline
\hline 
{\scriptsize{}Razão do Débito Técnico} & {\scriptsize{}0,7\%} & {\scriptsize{}0,7\%} & {\scriptsize{}0,7\%} & {\scriptsize{}0,7\%} & {\scriptsize{}0,7\%} & {\scriptsize{}0,7\%}\tabularnewline
\hline 
{\scriptsize{}Classificação de Manutenibilidade} & {\scriptsize{}A} & {\scriptsize{}A} & {\scriptsize{}A} & {\scriptsize{}A} & {\scriptsize{}A} & {\scriptsize{}A}\tabularnewline
\hline 
\end{tabular}
\par\end{centering}
\centering{}\caption{Avaliação das versões}
\label{Avalia=0000E7=0000E3o}
\end{table}

A Tabela \ref{Avalia=0000E7=0000E3o} apresenta um histórico da evolução
do Linux Kernel ao longo de 6 versões, da versão 3.19 à 4.4. As métricas
apresentadas são as mesmas descritas na Tabela \ref{M=0000E9tricasSonarQube}.

Como pode ser observado na Tabela \ref{Avalia=0000E7=0000E3o}, o
número de linhas cresce a cada lançamento. Pode-se supor que o constante
desenvolvimento de novos recursos, como novos \emph{drivers }de dispositivos
e seja responsável por tal aumento. O valor absoluto do Débito Técnico
também parece crescer a cada lançamento, tendo a ultima versão analisada
o maior volume encontrado. A primeira vista, o valor de débito técnico
parece bastante elevado, porém, levando em conta o número de linhas
do projeto e estimando um custo de desenvolvimento para cada linha,
assim como oferecido na Tabela \ref{M=0000E9tricasSonarQube}, podemos
avaliar o Débito Técnico total do projeto como baixo. Por tal razão,
o projeto do Linux Kernel recebe classificação de manutenibilidade
A, a mais alta dentro da escala estabelecida pelo SonarQube.

Diante destes resultados, pode-se supor que a manutenibilidade do
projeto foi mantida graças a estrutura organizacional do projeto.
Mantendo orientações de estilo de desenvolvimento e garantindo que
elas sejam cumpridas através do seu processo de revisão, os desenvolvedores
mantém o controle da forma como o código evolui. 


