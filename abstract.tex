Open-source software is a integral part of modern society, influencing a wide range of fields such as network infrastructure, end-user applications, programming languages and so forth. Although widely used, such systems are poorly understood and little is known about their evolution. We analyze the evolution of the Apache Hadoop project, since its creation to its most recent version, trying to validate Lehman law's of software evolution. Our results indicate that only the laws Continuous Change, Self-Regulation, and Continuous Growth are valid, while the laws Increasing Complexity, Conservation of Organizational Stability, Conservation of Familiarity, Decreasing Quality and Feedback System could not be confirmed. These results suggest that Lehman's Laws of software evolution may not describe well Open-Source software systems, indicating that more research in the area is required to properly understand the phenomenon.