O Linux Kernel é um maior projeto público de software atual. Um dos desafios do projeto é manter o código em boas condições de manutenibilidade, ou seja, a capacidade de realizar alterações no código fonte. Umas das formas formas de mensurar a manutenibilidade do código é analisar seu acoplamento global, ocorrências de compartilhamento de dados globais por dois módulos. Neste contexto, trabalhos anteriores analisaram 365 versões do Linux kernel contando o número de ocorrências de acoplamento global entre 17 módulos e todos os outros módulos, e verificaram que tais ocorrências cresciam de forma exponencial com o número da versão. Os autores afirmam que caso o software não fosse reestruturado visando minimizar o número de acoplamentos globais, o software tornaria-se difícil de ser mantido com o tempo. Este trabalho visa analisar a manutenibilidade de 6 versões de lançamento estáveis do Linux kernel com a finalidade de verificar o quão difícil tem sido manter seu código. Para tanto, o método SQALE foi utilizado para realizar a avaliação da manutenibilidade . Todas as versões obtiveram classificação de manutenibilidade A, indicando que o estado de manutenibilidade do projeto é saudável.