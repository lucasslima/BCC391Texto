Sistemas de \textit{software Open-Source} são parte integral do mundo moderno, atuando em áreas diversas como infraestrutura de rede, aplicações para o usuário final, linguagens de programação, entre outros. Apesar de muito utilizados, tais sistemas são pouco compreendidos e pouco se sabe de sua evolução. Neste trabalho, analisamos a evolução do \textit{Apache Hadoop}, desde sua criação até sua versão mais recente em uma tentativa de verificar se as Leis de Lehman sobre evolução de \textit{software}. Nossos resultados indicam que apenas as leis Mudança Contínua, Auto-Regulação e Crescimento Contínuo são válidas, enquanto as leis Complexidade Crescente, Conservação da Estabilidade Organizacional, Conservação da Familiaridade, Qualidade Descrescente e Sistema de Realimentação não puderam ser confirmadas. Tais resultados indicam que as Leis de Lehman podem não descrever bem sistemas de \textit{software Open-Source}, indicando que mais pesquisa na área é necessária.