% \iffalse meta-comment
%
% Copyright (C) 2004,2005 by Vilar Camara Neto and Eduardo Freire Nakamura.
% adapted for UFOP by David Menotti
%
% This file may be distributed and/or modified under the
% conditions of the LaTeX Project Public License, either
% version 1.2 of this license or (at your option) any later
% version.  The latest version of this license is in:
%
%     http://www.latex-project.org/lppl.txt
%
% and version 1.2 or later is part of all distributions of
% LaTeX version 1999/12/01 or later.
%
% Currently this work has the LPPL maintenance status "maintained".
%
% The Current Maintainer of this work is Vilar Camara Neto.
%
% This work consists of the files ufopthesis.dtx and
% ufopthesis.ins and the derived file ufopthesis.cls.
%
% \fi
%
% \iffalse
%<*driver>
\ProvidesFile{ufopthesis.dtx}
%</driver>
%<class>\NeedsTeXFormat{LaTeX2e}[1999/12/01]
%<class>\ProvidesClass{ufopthesis}
%<*class>
  [2005/07/24 v0.13 DCC/ufop class for thesis/dissertations and proposals]
%</class>
%<*driver>
\documentclass{ltxdoc}
\usepackage[latin1]{inputenc}
\usepackage[T1]{fontenc}
\EnableCrossrefs
\CodelineIndex
\RecordChanges
\begin{document}
\DocInput{ufopthesis.dtx}
\end{document}
%</driver>
% \fi

% \CheckSum{1248}


% \CharacterTable
%  {Upper-case    \A\B\C\D\E\F\G\H\I\J\K\L\M\N\O\P\Q\R\S\T\U\V\W\X\Y\Z
%   Lower-case    \a\b\c\d\e\f\g\h\i\j\k\l\m\n\o\p\q\r\s\t\u\v\w\x\y\z
%   Digits        \0\1\2\3\4\5\6\7\8\9
%   Exclamation   \!     Double quote  \"     Hash (number) \#
%   Dollar        \$     Percent       \%     Ampersand     \&
%   Acute accent  \'     Left paren    \(     Right paren   \)
%   Asterisk      \*     Plus          \+     Comma         \,
%   Minus         \-     Point         \.     Solidus       \/
%   Colon         \:     Semicolon     \;     Less than     \<
%   Equals        \=     Greater than  \>     Question mark \?
%   Commercial at \@     Left bracket  \[     Backslash     \\
%   Right bracket \]     Circumflex    \^     Underscore    \_
%   Grave accent  \`     Left brace    \{     Vertical bar  \|
%   Right brace   \}     Tilde         \~}
%
% \changes{v0.13}{2005/07/24}{The approval sheet now shows the broken title.}
% \changes{v0.12}{2005/07/21}{Bugs: arabic numbering in front matter (with
%   ``easy way'') and no header in main matter (with command
%   \string\ufopthesis).}
% \changes{v0.11}{2005/06/21}{Typo: \string\ufopthesis was previously named
%   \string\ufophesis; Front matter header/footer reformat; page numbering
%   starts right after cover/title pages; \string\mainmatter is now automatic
%   (see `nomainmatter' class option)}
% \changes{v0.10}{2005/03/12}{Cover/title page reformat; added `centertitles'
%   class option}
% \changes{v0.09}{2005/03/10}{Extended the Portuguese/non-Portuguese
%   integration}
% \changes{v0.08}{2005/03/09}{Changed date format and document structure,
%   especially to provide more integration between Portuguese and
%   non-Portuguese parts}
% \changes{v0.07}{2005/03/09}{Changed ``Thesis proposal'' to ``Thesis
%   project''}
% \changes{v0.06}{2005/03/07}{Corrected the bibliography header placement}
% \changes{v0.05}{2005/02/22}{Added bibliography support}
% \changes{v0.04}{2005/01/15}{Added `memhfixc' package auto-loading}
% \changes{v0.03}{2005/01/15}{`twoside' option is working now}
% \changes{v0.02}{2005/01/13}{Corrected small formatting bugs}
% \changes{v0.01}{2004/12/09}{Initial version (first draft)}
%
% \GetFileInfo{ufopthesis.dtx}
%
% \DoNotIndex{\#,\$,\%,\&,\@,\\,\{,\},\^,\_}
% \DoNotIndex{\`~}
% \DoNotIndex{\ }
% \DoNotIndex{\z@,\m@ne,\@ne,\@ifpackageloaded,\@ifundefined,\@temp}
% \DoNotIndex{\@undefined}
% \DoNotIndex{\addtocount,\advance,\AtBeginDocument,\begin,\bgroup}
% \DoNotIndex{\begingroup,\bibname,\c,\catcode,\closein,\centering}
% \DoNotIndex{\chapter,\chaptermark,\checkandfixthelayout,\cleardoublepage}
% \DoNotIndex{\contentsname,\copypagestyle,\csname,\date,\DeclareOption}
% \DoNotIndex{\define@key,\doublespace,\egroup,\end,\endcsname,\ExecuteOptions}
% \DoNotIndex{\expandafter,\fi,\frontmatter,\gdef,\height,\hspace}
% \DoNotIndex{\if,\if@twoside,\ifnum,\ifx,\includegraphics,\input}
% \DoNotIndex{\languagename,\Large,\large,\leftmark,\let,\listfigurename}
% \DoNotIndex{\listoffigures,\listoftables,\listtablename,\LoadClass}
% \DoNotIndex{\loop,\mainmatter,\makeevenhead,\makeheadrule,\makeoddfoot}
% \DoNotIndex{\makeoddhead,\MakeUppercase,\markboth,\markright,\maxsecnumdepth}
% \DoNotIndex{\message,\newcommand,\newcounter,\newif,\normalrulethickness}
% \DoNotIndex{\normalsize,\onehalfspace,\pagestyle,\par,\PassOptionsToClass}
% \DoNotIndex{\ProcessOptions,\protect,\quad,\raisebox,\relax,\renewcommand}
% \DoNotIndex{\repeat,\RequirePackage,\rightmark,\sectionmark,\setcounter}
% \DoNotIndex{\setkeys,\setlrmarginsandblock,\setulmarginsandblock}
% \DoNotIndex{\singlespace,\stretch,\tableofcontents,\textbf,\textit,\textsc}
% \DoNotIndex{\textwidth,\thechapter,\thepage,\thesection,\title}
% \DoNotIndex{\closeout,\day,\def,\edef,\else,\empty,\endgroup,\vspace}
%
% \title{The \textsf{ufopthesis} class\thanks{This document corresponds to
% \textsf{ufopthesis}~\fileversion, dated~\filedate.}}
% \author{Vilar Camara Neto\qquad\qquad Eduardo Nakamura \\
% \texttt{\{neto,nakamura\}@dcc.ufop.br}}
% \date{July 24, 2005}
%
% \maketitle
%
% \StopEventually
%
% \section{Introduction}
%
% This file describes the class file |ufopthesis|, which is an unofficial class
% to typeset master thesis and Ph.D.\@ dissertations for the Federal University
% of Minas Gerais (ufop). This work is widely based on the original class
% |ufopthes| made by student Eduardo Freire Nakamura, which is being used by
% many other students to format their documents.
%
% As stated, this is an \emph{unofficial} class! At present, it's not endorsed
% nor being supported in any way by the institution. As usual in these cases,
% this class is provived in an ``as is'' basis, without guarantees of any type.
% If you use |ufopthesis| for your document, please discuss the formatting with
% your advisor and make the suggested adaptations/corrections/improvements. If
% you think that the changes you made are interesting to other students, please
% share the new version! But don't forget to increment the version number and
% describe what did you do.
%
%
% \section{How to use it}
%
% This section provides information necessary to use the |ufopthesis|
% class. \emph{Please read it!} This is important. I've told you.
%
% Note that |ufopthesis| is based on |memoir|, which is a highly configurable
% class. You may have to install it; please refer to Section
% \ref{sec:memoirtips}, where I provide a very short description about
% it. There are some important tips too.
%
%
% \subsection{Important information to users of previous versions}
%
% If you've been using a previous version of the |ufopthesis| class and are now
% upgrading to the current version, this subsection lists the structural
% changes to the |ufopthesis| class from previous versions that affect the user
% (you), like changes in the syntax of commands or options. If you are using
% the |ufopthesis| class for the first time, this subsection isn't relevant to
% you.
%
% \subsubsection{Changes introduced in version 0.11}
%
% \textbf{Important!} Previous versions have an unnoticed typo! The macro
% |\ufopthesis| was named |\ufophesis| (i.e., without the \textbf{t}). This was
% corrected from version 0.11 on. When upgrading, \emph{please} correct it in
% your document.
%
% The |\mainmatter| command is now called automatically by the |\ufopthesis|
% command, unless the |nomainmatter| class option is used.
%
% The header/footer format was changed: the section title doesn't appear
% anymore, because the header got overloaded with much text. (Sometimes the
% chapter and section titles overlapped.)
%
% Also, the page numbering starts right after cover/title pages. (In previous
% versions, it started only in the table of contents.)
%
% \subsubsection{Changes introduced in version 0.10}
%
% The format of cover and title pages were changed. In fact, the difference
% isn't only the presence/absence of the descriptive text: the author position
% also changes.
%
% \subsubsection{Changes introduced in version 0.09}
%
% The options |portuguesetitle|, |portuguesebrokentitle|,
% |portugueseuniversity| and |portuguesecourse| were created. They're only
% important if you're writting the document in non-Portuguese. Also, the field
% |department| ceased to exist (this information isn't used).
%
% \subsubsection{Changes introduced in version 0.08}
%
% Mainly, the format of the date provided by the |date=|\meta{date} option (to
% the |\ufopthesis| command) or by the |\date|\marg{date} command was changed.
% Instead of providing the textual date (like ``10 de janeiro de 2005'' or
% ``January 10, 2005''), you must now provide the date as |yyyy-mm-dd| or
% |yyyy-mm|. See Section~\ref{sec:ufopthesis} for more details.
%
%
% \subsection{Class options}
%
% \newcommand{\defmark}{$\triangleright$}
% \newcommand{\sep}{$\mid$}
%
% When using the |\documentclass| command, you can give some options to
% customize the document appearance. The available options are listed below;
% the symbol \defmark{} marks the default one.
%
% \begin{itemize}
% \item |msc| \sep{} \defmark|phd|\par
%   Defines if this document is a master thesis or a Ph.D.\@ dissertation.
% \item |proposal| or |project|\par
%   If one of these options is given, the document is a dissertation proposal
%   or a thesis project. (No matter what option is given, the effect is the
%   same: there is no ``dissertation project'' or ``thesis proposal''.)
% \item |single| \sep{} \defmark|onehalf| \sep{} |double|\par
%   Defines the default line spacing: single, 1\,\textonehalf{} or double.
% \item |centertitles|\par
%   By default, the preamble ``chapter'' texts (Abstract, Acknowledgment, Table
%   of Contents, List of Figures/Tables) are left-aligned. If this option is
%   given, these texts will be centered. Also affects the bibliography title if
%   the |\ufopbibliography| command is used. This doesn't affect the alignment
%   of any |\chapter| or |\chapter*| used throughout the document.
% \item \defmark|showcover| \sep{} |hidecover|\par
%   Enables or disables the cover page.
% \item \defmark|showtitle| \sep{} |hidetitle|\par
%   Enables or disables the title page.
% \item \defmark|showapproval| \sep{} |hideapproval|\par
%   Enables or disables the approval sheet.
% \item \defmark|showabstract| \sep{} |hideabstract|\par
%   Enables or disables the abstract page(s).
% \item \defmark|showdedication| \sep{} |hidededication|\par
%   Enables or disables the dedication page.
% \item \defmark|showack| \sep{} |hideack|\par
%   Enables or disables the acknowledgment page.
% \item \defmark|showtoc| \sep{} |hidetoc|\par
%   Enables or disables the Table of Contents.
% \item \defmark|showlof| \sep{} |hidelof|\par
%   Enables or disables the List of Figures.
% \item \defmark|showlot| \sep{} |hidelot|\par
%   Enables or disables the List of Tables.
% \item \defmark|showall| \sep{} |hideall|\par
%   Enables or disables all pages, as if all |show|* options were given. You
%   can use the |hideall| option to speed up the compilation time during the
%   document development phase, and switch to |showall| only to produce the
%   final version.
% \item |nomainmatter|\par
%   The |\ufopthesis| macro calls the |\mainmatter| command after everything is
%   done. (|\mainmatter| resets page numbering and alters its format to arabic
%   --- 1, 2, 3, etc..) If you want to manually put extra material in the front
%   matter --- i.e., to put pages while maintaining the roman page numbering,
%   like a List of Algorithms or something similar ---, then you must use the
%   |nomainmatter| option. In this case, \emph{don't forget} to use the
%   |\mainmatter| command right before the first chapter. If you forget it, all
%   page numbers will be in roman format.
% \end{itemize}
%
%
% \subsection{The very easy way}\label{sec:ufopthesis}
%
% All introductory pages --- cover, main title, approval sheet, abstracts,
% dedication, table of contents, list of figures and list of tables --- can be
% generated by one single command. No kidding. The magic command is
% |\ufopthesis|, which receives all the information via
% \meta{key}|=|\meta{value} pairs. The command call will look like:
%
% \begin{quote}
% \begin{raggedright}
% |\ufopthesis{|\\
% |  title={The Thesis Title},|\\
% |  author={Vilar Fiuza da Camara Neto},|\\
% |  university={Universidade Federal de Minas Gerais},|\\
% |  |\emph{other commands\ldots}\\
% |}|
% \end{raggedright}
% \end{quote}
%
% What you do need now is a list describing every key/value pair that can
% happen inside the |\ufopthesis| argument. Well, here it is:
%
% \begin{itemize}
% \item |title|\marg{title}\par
%   Defines the thesis/dissertation title.
% \item |brokentitle|\marg{title}\par
%   Defines the title that appears in the first two pages (cover and title
%   page). This is optional; if you don't define \texttt{brokentitle}, the
%   \texttt{title} text is used. The \texttt{brokentitle} option is provided
%   because you can specify where are the line breaks (with double backslashes,
%   |\\|), in the case of a long title that spans more than one line. Please
%   see the description of the |\brokentitle| macro in page
%   \pageref{macro:brokentitle} for more details.
% \item |author|\marg{name}\par
%   Defines the author name.
% \item |university|\marg{name}\par
%   Defines the university name.
% \item |course|\marg{name}\par
%   Defines the course name.
% \item |portuguesetitle|\marg{title}\\
%   |portuguesebrokentitle|\marg{title}\\
%   |portugueseuniversity|\marg{name}\\
%   |portuguesecourse|\marg{name}\par
%   If you're writting your document in non-Portuguese, you need to provide the
%   Portuguese versions of the title, university and name. These options
%   provide the way to go. They work exactly like their counterparts without
%   the |portuguese| prefix.
% \item |address|\marg{address}\par
%   Well, it's not the complete address, just the city and state to appear at
%   the bottom of the title pages.
% \item |date|\marg{date}\par
%   Defines the document date in short format, \emph{always} as |yyyy-mm-dd| or
%   |yyyy-mm| (the day isn't obligatory).\footnote{The format of the date has
%   changed since version 0.08.} Note the order (year-month-day), the hyphen as
%   the separator, and the use of 4-digit for the year.
% \item |logo|\marg{file}\par
%   Specifies a graphic file that contains the university logo. This appears at
%   the top of the approval sheet.
% \item |advisor|\marg{name}\marg{title}\marg{institution}\par
%   Defines the advisor data. \meta{title} is something like
%   ``\texttt{Ph.\~{}D.}'' or ``\texttt{M.\~{}Sc.}'' (please use a tilde after
%   the first dot instead of a space --- this will look better).
%   \meta{institution} is the university from where the advisor comes. If you
%   use a single dot as the \meta{institution} argument --- e.g.,
%   \texttt{advisor=}\marg{name}\marg{title}|{.}| ---, it's copied from the
%   \texttt{university} value. The advisor name will appear in the approval
%   sheet.
% \item |coadvisor|\marg{name}\marg{title}\marg{institution}\par
%   Defines the co-advisor data. The arguments are the same as
%   \texttt{advisor}.
% \item |member|\oarg{relationship}\marg{name}\marg{title}\marg{institution}%
%   \par
%   Defines another comitee member. You can use this key several times, one for
%   each member. The arguments are the same as \texttt{advisor}, except that
%   there's an optional \meta{relationship}: this is a text that will appear
%   after the member name, in the approval sheet, \emph{always} in Portuguese.
%   It's normally not used.
% \item |abstract|\marg{filename}\par
%   Defines a \texttt{.tex} file containing the abstract text in the native
%   language. Don't use the |\section| command in this file.
% \item |englishabstract|\marg{filename}\par
%   Defines a \texttt{.tex} file containing the abstract text in English
%   language. This option should be used only if your document is written in
%   Portuguese and you want to provide an English-version of the abstract.
%   Don't use the |\section| command in this file.
% \item |portugueseabstract|\marg{filename}\par
%   Defines a \texttt{.tex} file containing the abstract text in Portuguese
%   language. This option should be used only if your document isn't written in
%   Portuguese. Don't use the |\section| command in this file.
% \item |dedication|\marg{filename}\par
%   Defines a \texttt{.tex} file containing the dedication text. Don't use the
%   |\section| command in this file.
% \item |ack|\marg{filename}\par
%   Defines a \texttt{.tex} file containing the acknowledgment text. Don't use
%   the |\section| command in this file.
% \end{itemize}
%
% Here's a complete example (for a document written in Portuguese):
%
% \begin{quote}
% \begin{verbatim}
% \ufopthesis{
%   title={Protocolo para Verifica��o de Erros em Redes Totalmente
%     Confi�veis},
%   brokentitle={Protocolo para Verifica��o de Erros\\em Redes
%     Totalmente Confi�veis},
%   author={Vilar Fiuza da Camara Neto},
%   university={Universidade Federal de Minas Gerais},
%   course={Ci�ncia da Computa��o},
%   address={Belo Horizonte},
%   date={2005-01-30},
%   logo={img/brasao},
%   advisor={Adamastor Pompeu Set�bal}{Ph.~D.}{.},
%   coadvisor={Romualdo Freij� Quaresma}{Ph.~D.}{.},
%   member=[Co-revisor]{Aristolfo Hostetten}{M.~Sc.}{Universidade
%     Federal do Paran�},
%   member={Isa�as Fortunato de Oliveira Neto Jr.}{M.~Sc.}
%     {Universidade Estadual de S�o Jo�o do Pirapora},
%   member={Deodato Al�pio Doberval Etrusco}{M.~Sc.}
%     {Universidade de Ves�vio, Pomp�ia},
%   abstract={resumo.tex},
%   englishabstract={abstract.tex},
%   dedication={dedicatoria.tex},
%   ack={agradecimentos.tex},
% }
%\end{verbatim}
% \end{quote}
%
% Note that here we used the |abstract| and |englishabstract| options to inform
% the files containing the Portuguese and English texts, respectively. If
% you're typing a document in English, this must change: you have to use
% |abstract| to provide the English text and |portugueseabstract| to provide
% the Portuguese one. And you must provide |portuguesetitle|,
% |portuguesebrokentitle| (this is optional, as |brokentitle|),
% |portugueseuniversity| and |portuguesecourse|.
%
% The following example illustrates the structure of |\ufopthesis| for a
% document written in non-Portuguese:
%
% \begin{quote}
% \begin{verbatim}
% \ufopthesis{
%   title={Protocol for Error-Verification inside Totally Error-Free
%     Networks},
%   brokentitle={Protocol for Error-Verification inside\\Totally
%     Error-Free Networks},
%   author={Vilar Fiuza da Camara Neto},
%   university={Federal University of Minas Gerais},
%   course={Computer Science},
%   portuguesetitle={Protocolo para Verifica��o de Erros em Redes
%     Totalmente Confi�veis},
%   portuguesebrokentitle={Protocolo para Verifica��o de Erros\\
%     em Redes Totalmente Confi�veis},
%   portugueseuniversity={Universidade Federal de Minas Gerais},
%   portuguesecourse={Ci�ncia da Computa��o},
%   address={Belo Horizonte},
%   date={2005-01-30},
%   logo={img/brasao},
%   advisor={Adamastor Pompeu Set�bal}{Ph.~D.}{.},
%   coadvisor={Romualdo Freij� Quaresma}{Ph.~D.}{.},
%   member=[Co-revisor]{Aristolfo Hostetten}{M.~Sc.}{Universidade
%     Federal do Paran�},
%   member={Isa�as Fortunato de Oliveira Neto Jr.}{M.~Sc.}
%     {Universidade Estadual de S�o Jo�o do Pirapora},
%   member={Deodato Al�pio Doberval Etrusco}{M.~Sc.}
%     {Universidade de Ves�vio, Pomp�ia},
%   portugueseabstract={resumo.tex},
%   abstract={abstract.tex},
%   dedication={dedicatoria.tex},
%   ack={agradecimentos.tex},
% }
%\end{verbatim}
% \end{quote}
%
%
% \subsection{The easy way}
%
% There's another way to build the pages: you can forget the magic
% |\ufopthesis| macro and call a series of commands that defines each
% information, and then some macros that build each page. It's a bit more
% arduous, but you have more control over page ordering.
%
% The commands to define the document information have the same names and
% arguments as the key/value pairs shown in the previous section: you call
% |\title|\marg{title}, |\author|\marg{author}, and so on. The only exception
% is the command
% |\addtocomitee|\oarg{relationship}\marg{name}\marg{title}\marg{institution},
% which corresponds to the |member| key (a command called |\member| sounds too
% generic for me).
%
% After defining all information, you must call a macro to build each page. The
% macros are shown in the following table:
%
% \begin{itemize}
% \item |\makecoverpage| \par
%   Creates the cover page, containing the author name, document title, address
%   and date.
% \item |\maketitlepage| \par
%   Creates the title page. It's the same as the cover page, but includes a
%   short description of the document.
% \item |\makeapproval| \par
%   Creates the approval sheet. This page lists every comitee member, leaving
%   some inter-space to the signatures. Currently, this page is made in
%   Portuguese, no matter what the document language is.
% \item |\includeabstract|\oarg{title}\marg{filename} \par
%   Includes an external file containing the abstract text. \meta{title} is the
%   abstract title, which defaults to ``Abstract'' translated to the current
%   language. (It may be used to include an alternative abstract written in a
%   foreign language.)
% \item |\includededication|\marg{filename} \par
%   Includes an external file containing the dedication text.
% \item |\includeack|\marg{filename} \par
%   Includes an external file containing the acknowledgment text.
% \item |\tableofcontents| \par
%   Builds a table of contents.
% \item |\listoffigures| \par
%   Builds a list of figures.
% \item |\listoftables| \par
%   Builds a list of tables.
% \end{itemize}
%
% After these commands, you must use the |\mainmatter| macro to reset page
% numbering and configure it to be in arabic format (1, 2, 3, etc.).
%
%
% \subsection{The hard way}
%
% Not yet implemented\ldots{} :-)
%
%
% \subsection{The bibliography}
%
% You can use any bibliography system/style: the |ufopthesis| doesn't obligate
% the adoption of any particular system.
%
% However, if you're using the Bib\TeX{} system (to manage |.bib| files and
% compile them into |.bbl| files), you may be wondering about what style to use
% (i.e., what should be the argument to |\bibliographystyle{}|?). Since there's
% no official bibliography style adopted by the university (as far as we know),
% here we provide a style which is heavily based on the traditional |apalike|
% style: the |ufop| style (file |ufop.bst|). As the |ufopthesis.cls|, you must
% put |ufop.bst| in a directory visible to the Bib\TeX{} system.
%
% The |ufopthesis| class also provides a command to automatically generate the
% bibliography (see below).
%
% Again, if you adopt the provided style, the main rule still applies here:
% please discuss the formatting with your advisor and make the suggested
% adaptations/corrections/improvements.
%
% The easiest form of making the bibliography is using the |\ufopbibliography|
% command. That's just it: there's no need to use |\bibliographystyle| nor
% |\bibliography| commands. The format is:
%
% \begin{quote}
% |\ufopbibliography|\oarg{options}\marg{bibliography-file}
% \end{quote}
% where \meta{bibliography-file} is the name of your |.bib| file without the
% extension. So, if your reference database is stored in a file named
% |myreferences.bib|, you must call |\ufopbibliography{myreferences}|.
%
% There are a bunch of possible \meta{options}, mostly related to the
% formatting of the bibliography. The available options are:
%
% \begin{itemize}
% \item |noauthorrepeat|\par
%   If you have a sequence of bibliographic items written by the same authors,
%   this options makes them to appear only once, in the first entry: the
%   following ones show a triple-em-dash (e.g., ``---------'') in place of the
%   names. (This can be customized: see below.)
% \item |noauthorrepstring|\par
%   When using the |noauthorrepeat| option, you can customize what appears in
%   place of repeating authors. The default is a triple-em-dash (specified by a
%   sequence of 9 hyphens). This option allows the customization. Example:
%   |noauthorrepstring={Idem}| causes the string ``Idem'' to appear.
% \item |nobreakitems|\par
%   This option forbids a page break inside any bibliographic entry, i.e., the
%   text lines of each item are kept together in the same page.
% \item |bibauthorand|\par
%   Defines the text that appears before the last author in the bibliography.
%   The default is ``and'' (English) or ``e'' (Portuguese). Example: to specify
%   an ampersand as the text, use |bibauthorand=\&|.
% \item |citeauthorand|\par
%   Defines the text that appears before the last author in the citations
%   (e.g., when using |\cite|\marg{entry} in the main text).  The default is
%   ``and'' (English) or ``e'' (Portuguese). Example: to specify an ampersand
%   as the text, use |citeauthorand=\&|.
% \item |authorformat|\par
%   Defines the format of the author list. The default is none. Example: to put
%   the author list in small caps, use |authorformat=\scshape|.
% \item |titleformat|\par
%   Defines the format of the title. The default is none. Example: to put the
%   title in italics, use |titleformat=\itshape|.
% \item |btitleformat|\par
%   Defines the format of the booktitle (for |@incollection| and
%   |@inproceedings| entries). The default is |\itshape| (italics).
% \item |booktitleformat|\par
%   Defines the format of the booktitle (for |@book|, |@inbook|, |@manual|,
%   |@phdthesis|, and |@proceedings| entries). The default is |\itshape|
%   (italics).
% \item |journalformat|\par
%   Defines the format of the journal name (for |@article| entries). The
%   default is |\itshape| (italics).
% \end{itemize}
%
% By default, certain things are automatically localized (currently, only to
% English and Brazilian Portuguese languages): e.g., if you have an article
% made by authors ``Michael J. Brooks and Berthold K. P. Horn'', they'll appear
% as ``Brooks and Horn'' if your thesis is being written in English, or
% ``Brooks e Horn'' if written in Portuguese. To use this default behaviour,
% you don't need to specify any option: just provide the correct language to
% the |babel| package.
%
% Finally, I \emph{strongly} recommend the use of the |natbib| package, to make
% more professional-looking references. It's more than just a matter of
% personal preference: I've seen advisors asking students to adopt |natbib|.
%
%
% \subsection{Tips about the \texttt{memoir} class}\label{sec:memoirtips}
%
% As previously stated, this class is based on |memoir|, which must be present
% on your system. If it isn't --- for instance, it complains about the absence
% of the file |memoir.cls| --- you have to install it. If you have a MiK\TeX{}
% distribution, this is quite easy: it's a matter of running the MiK\TeX{}
% Package Manager, selecting ``memoir'' from the list and installing. You may
% need administrative privileges to do so: consult the system administrator.
%
% Under Linux, the installation is not automated, but not hard at all. Download
% all files from the corresponding directory on CTAN: the present working link
% is
%
% \begin{quote}
% |http://www.ctan.org/tex-archive/macros/latex/contrib/memoir/|
% \end{quote}
%
% When the download is complete, open a terminal, |cd| to where the downloaded
% files are and run
%
% \begin{quote}
% |latex memoir.ins|
% \end{quote}
%
% which will create the so-wanted |memoir.cls| file. Then run |latex| again,
% this time over |memoir.dtx|:
%
% \begin{quote}
% |latex memoir.dtx|
% \end{quote}
%
% This will create the documentation files. Now create a local |texmf|
% directory under your home:
%
% \begin{quote}
% |mkdir ~/texmf/tex/latex/memoir|
% \end{quote}
%
% and copy into it all |*.cls|, |*.sty| and |*.clo| files. Finally, run
% |texhash|.  That's it, you've done!
%
% Anyway, it's never a bad idea to read the |README| file. This section was
% written based on information found inside |README| and at
% |http://www.tex.ac.uk/| |cgi-bin/texfaq2html?label=instpackages|.
%
%
% \subsection{Few tips to speed your work}
%
% It's likely that you'll want the following lines in the preamble of your
% |.tex| file:
%
% \begin{quote}
% \begin{verbatim}
% \usepackage[brazil]{babel}
% \usepackage[latin1]{inputenc}
% \usepackage[T1]{fontenc}
%\end{verbatim}
% \end{quote}
%
% If you want graphics (who doesn't?):
%
% \begin{quote}
% \begin{verbatim}
% \usepackage[dvips]{graphicx}
%\end{verbatim}
% \end{quote}
%
% Substitute |dvips| for |dvipdf| or |dvipdfm| if you're using one of these
% systems.
%
% If you want to use |hyperref| to provide hyperlinks to the generated |.pdf|
% file:
%
% \begin{quote}
% \begin{verbatim}
% \usepackage[dvips,
%   a4paper,
%   portuguese,
%   bookmarks=true,
%   bookmarksnumbered=true,
%   linktocpage,
%   ]{hyperref}
%\end{verbatim}
% \end{quote}
%
%
% \section{Source code}
%
% This section describes the |ufopthesis.cls| source code.
%
% \subsection{Prerequisites and general macros}
%
% The |ufopthesis| requires some packages, so let's include them:
%    \begin{macrocode}
\RequirePackage{keyval}
\RequirePackage{setspace}
%    \end{macrocode}


% The class options are handled by |\newif|s, which are declared and
% initialised:
%
%    \begin{macrocode}
%% General options:

\newif\if@phd                 % phd dissertation option
\newif\if@msc                 % master thesis option
\newif\if@proposal            % proposal/project option
\newif\if@mono                % monograph option
\newif\if@showcover           % Show cover page?
\newif\if@showtitle           % Show title page?
\newif\if@showapproval        % Show approval sheet?
\newif\if@showabstract        % Show abstract page(s)?
\newif\if@showdedication      % Show dedication page?
\newif\if@showack             % Show acknowledgment page?
\newif\if@showtoc             % Show Table of Contents?
\newif\if@showlof             % Show List of Figures?
\newif\if@showlot             % Show List of Tables?
\newif\if@extraporttitlepage  % Show also a title page in Portuguese?
\newif\if@putmainmatter       % Put \mainmatter after TOC/LOF/LOT?

\def\ufop@showall{%
  \@showcovertrue
  \@showtitletrue
  \@showapprovaltrue
  \@showabstracttrue
  \@showdedicationtrue
  \@showacktrue
  \@showtoctrue
  \@showloftrue
  \@showlottrue
}

\def\ufop@hideall{%
  \@showcoverfalse
  \@showtitlefalse
  \@showapprovalfalse
  \@showabstractfalse
  \@showdedicationfalse
  \@showackfalse
  \@showtocfalse
  \@showloffalse
  \@showlotfalse
}

\@proposalfalse
\ufop@showall
\@putmainmattertrue

%% Bibliography options:

\newif\if@bibnorepauthor  % Substitute repeating names by a long line?
\newif\if@nobreakitems    % Dissalow page breaks between lines of each item?

\@bibnorepauthorfalse
\@nobreakitemsfalse
%    \end{macrocode}

% \DescribeMacro{\BreakableUppercase}%
% |\BreakableUppercase| provides a hack to |\MakeUppercase| which allows the
% ocurrence of linebreaks (|\\|) in the argument. As |\MakeUppercase|, this
% macro is called with one mandatory argument: |\BreakableUppercase|
% \marg{text}.
%
%    \begin{macrocode}
\def\BreakableUppercase#1{%
  \bgroup
  \let\ufop@prevdbs=\\%
  \def\\{\protect\ufop@prevdbs}%
  \MakeUppercase{#1}%
  \egroup
}
%    \end{macrocode}

% \DescribeMacro{\ufop@defspacing}%
% This macro stores the default document spacing, which is user-selectable
% via one of the following class options: |single|, |onehalf| or |double|.
%
%    \begin{macrocode}
\newcommand{\ufop@defspacing}{\onehalfspace}
%    \end{macrocode}

% \DescribeMacro{\ufop@redefchaptitlefont}%
% This macro redefines the formatting of the text produced by the |\chapter|
% command. Its sole objective is to redefine the |\chaptitlefont| (see the
% |memoir| manual) to add the |\centering| keyword if the |\documentclass|'s
% |centertitles| option is used.
%
%    \begin{macrocode}
\def\ufop@redefchaptitlefont{}
%    \end{macrocode}

% \DescribeMacro{\university}%
% \DescribeMacro{\course}%
% \DescribeMacro{\address}%
% This is a set of commands to define some information about the document. Some
% of them are mandatory, since the corresponding information appears somewhere
% in the auto-generated text. Each macro requires one argument. Call each one
% with the corresponding information in the preamble of the document (i.e.,
% before |\begin{document}|).
%
%    \begin{macrocode}
\newcommand*{\university}[1]{\gdef\@university{#1}}  % University name
\newcommand*{\course}[1]{\gdef\@course{#1}}          % Course name
\newcommand*{\shortcourse}[1]{\gdef\@shortcourse{#1}}          % Short Course name
\newcommand*{\address}[1]{\gdef\@address{#1}}        % Address
%    \end{macrocode}

% \DescribeMacro{\degree}%
% This macro specifies the degree title: it's not mandatory, since a default is
% provided --- a translated form of ``Master'' or ``Doctor'', according to the
% correspondent class option --- |[msc]| or |[phd]| --- and the current
% language, selected via package |babel|.
%
%    \begin{macrocode}
\newcommand*{\degree}[1]{\gdef\@degree{#1}}          % Degree
%    \end{macrocode}
%
% \DescribeMacro{\logo}%
% This defines a graphic file containing the picture used in the approval
% sheet.  You may include the path and extension. If this command isn't used,
% the approval sheet is built without a logo. Remember that the file format
% (extension) must be recognized by the \LaTeX{} system being used: for
% example, |.eps| files aren't compatible with |pdflatex|.
%
%    \begin{macrocode}
\newcommand*{\logo}[1]{\gdef\@logo{#1}}              % Logo
%    \end{macrocode}
%
% \DescribeMacro{\brokentitle}%
% \label{macro:brokentitle}%
% When the document title is long and spans more than one line in the cover
% page, the automatic line breaks may cause the text to be unbalanced between
% lines. For example, if your title is ``This is a long title for my thesis'',
% the text may be broken as follows:
%
% \begin{center}
% This is a long title for my\\thesis
% \end{center}
%
% In this case, you may use the macro |\brokentitle| \marg{text} to manually
% define the line breaks, e.g:
% |\brokentitle{This is a long title\\for my thesis}| to generate a more
% balanced aspect:
%
% \begin{center}
% This is a long title\\for my thesis
% \end{center}
%
% The use of |\brokentitle| isn't required, if the aspect based on the title
% provided by |\title| looks good enough.
%
%    \begin{macrocode}
\newcommand{\brokentitle}[1]{\gdef\@brokentitle{#1}} % Long broken title
%    \end{macrocode}
%
% \DescribeMacro{\title}%
% The original |\title| macro is redefined to provide a default for
% |\brokentitle|:
%
%    \begin{macrocode}
\renewcommand*{\title}[1]                            % Title
  {\gdef\@title{#1}\brokentitle{#1}}
%    \end{macrocode}

% \DescribeMacro{\portuguesetitle}%
% \DescribeMacro{\portuguesebrokentitle}%
% \DescribeMacro{\portugueseuniversity}%
% \DescribeMacro{\portuguesecourse}%
% For documents written in non-Portuguese, there are some macros to provide the
% related information in Portuguese:
%    \begin{macrocode}
\newcommand*{\portuguesetitle}[1]                    % Pt. title
  {\gdef\@portuguesetitle{#1}\portuguesebrokentitle{#1}}
\newcommand{\portuguesebrokentitle}[1]               % Pt. long broken title
  {\gdef\@portuguesebrokentitle{#1}}
\newcommand*{\portugueseuniversity}[1]               % Pt. university name
  {\gdef\@portugueseuniversity{#1}}
\newcommand*{\portuguesecourse}[1]                   % Pt. course
  {\gdef\@portuguesecourse{#1}}
\newcommand*{\portugueseshortcourse}[1]              % Pt. short course
  {\gdef\@portugueseshortcourse{#1}}

%    \end{macrocode}

% There are several commands that deal with class options:
%
%    \begin{macrocode}
\DeclareOption{mono}{\@phdfalse\@mscfalse\@monotrue}
\DeclareOption{msc}{\@phdfalse\@msctrue\@monofalse}
\DeclareOption{phd}{\@phdtrue\@mscfalse\@monofalse}
\DeclareOption{proposal}{\@proposaltrue}
\DeclareOption{project}{\@proposaltrue}

\DeclareOption{single}{\renewcommand{\ufop@defspacing}{\singlespace}}
\DeclareOption{onehalf}{\renewcommand{\ufop@defspacing}{\onehalfspace}}
\DeclareOption{double}{\renewcommand{\ufop@defspacing}{\doublespace}}

\DeclareOption{hideall}{\ufop@hideall}
\DeclareOption{hidecover}{\@showcoverfalse}
\DeclareOption{hidetitle}{\@showtitlefalse}
\DeclareOption{hideapproval}{\@showapprovalfalse}
\DeclareOption{hideabstract}{\@showabstractfalse}
\DeclareOption{hidededication}{\@showdedicationfalse}
\DeclareOption{hideack}{\@showackfalse}
\DeclareOption{hidetoc}{\@showtocfalse}
\DeclareOption{hidelof}{\@showloffalse}
\DeclareOption{hidelot}{\@showlotfalse}

\DeclareOption{showall}{\ufop@showall}
\DeclareOption{showcover}{\@showcovertrue}
\DeclareOption{showtitle}{\@showtitletrue}
\DeclareOption{showapproval}{\@showapprovaltrue}
\DeclareOption{showabstract}{\@showabstracttrue}
\DeclareOption{showdedication}{\@showdedicationtrue}
\DeclareOption{showack}{\@showacktrue}
\DeclareOption{showtoc}{\@showtoctrue}
\DeclareOption{showlof}{\@showloftrue}
\DeclareOption{showlot}{\@showlottrue}

\DeclareOption{nomainmatter}{\@putmainmatterfalse}

\DeclareOption{centertitles}{%
   \def\ufop@redefchaptitlefont{%
    \expandafter\renewcommand\expandafter{\expandafter\chaptitlefont
    \expandafter}\expandafter{\chaptitlefont\centering}%
  }%
}

\DeclareOption*{\PassOptionsToClass{\CurrentOption}{memoir}}
%    \end{macrocode}

% Default options follow:
%
%    \begin{macrocode}
\ExecuteOptions{phd,onehalf,showall}
\ProcessOptions
%    \end{macrocode}

% \DescribeMacro{\ufop@setdoctype}%
% This macro is defined to check boolean registers to adjust some internal
% registers --- namely, |\@degree| and |\@documenttype|. This macro is called
% within |\ufop@selectlanguage| macro, which is called at |\begin{document}|.
%
%    \begin{macrocode}
\def\ufop@setdoctype{%
  \if@phd % this is a PhD thesis
    \degree{\ufop@phd}
    \if@proposal
      \gdef\@documenttype{\ufop@phddocproj}%
      \ufop@set@phddocproj
    \else
      \gdef\@documenttype{\ufop@phddoc}%
      \ufop@set@phddoc
    \fi
  \else   % this is a MSc dissertation
    \degree{\ufop@msc}
    \if@proposal
      \gdef\@documenttype{\ufop@mscdocprop}%
      \ufop@set@mscdocprop
    \else
      \gdef\@documenttype{\ufop@mscdoc}%
      \ufop@set@mscdoc
    \fi
  \fi
}
%    \end{macrocode}

% Now there comes the page formatting commands. Here are the options regarding
% the page size, default font size, pagination, page margins, etc.. The entire
% work is based on class |memoir|, which greatly simplifies the layout
% configuration. See Section \ref{sec:memoirtips} about the use of |memoir|.
%
%    \begin{macrocode}
\LoadClass{memoir}

\setlrmarginsandblock{3cm}{2.5cm}{*}
\setulmarginsandblock{3cm}{3cm}{*}

\checkandfixthelayout
%    \end{macrocode}

% Index and topic numbering depth configuration:
%
%    \begin{macrocode}
\maxsecnumdepth{subsubsection}
\setcounter{tocdepth}{2}
\setcounter{secnumdepth}{4}
%    \end{macrocode}


% \subsection{Localization routines}
%
% \DescribeMacro{\ufop@langgroup}%
% Now comes the localization routines. We'll try to use the language passed to
% package |babel|; if this package isn't being used, the default language is
% |brazil| (not |english|, since this class is meant to be used primary in
% Brazil).
%
%    \begin{macrocode}
\def\ufop@langgroup{\@ifundefined{languagename}{brazil}{\languagename}}
%    \end{macrocode}

% \DescribeMacro{\ufop@deflanguage}%
% \DescribeMacro{\ufop@selectlanguage}%
% The pair of macros |\ufop@deflanguage|\marg{language} and
% |\ufop@selectlanguage| work together to set the translation table used to
% build the text in the final document. The first macro takes the language name
% as argument; the latter is the one that actually does the work.  Currently
% the available languages are |brazil| and |english|.
%
% This macro is automatically called with the language passed to |babel|. If
% the language isn't recognized, a warning is issued and the default |brazil|
% is adopted.
%
%    \begin{macrocode}
\def\ufop@deflanguage#1{%
  \@ifundefined{ufop@lang@#1}{%
    \ClassWarning{ufopthesis}{Language not defined: #1^^J
    Using default language `brazil'.^^J}%
    \def\ufop@langgroup{brazil}}
    {\def\ufop@langgroup{#1}}%
}
\def\ufop@selectlanguage{%
  \csname ufop@lang@\ufop@langgroup\endcsname\ufop@setdoctype
}
%    \end{macrocode}

% The following are auxiliary macros and |newif|s to help the translation
% routines.
%
%    \begin{macrocode}
\newif\if@ufopdateerror

\newcommand{\ufop@formatdate}[2][]{%
  \begingroup
    \def\ufop@templang{#1}%
    \ifx\ufop@templang\empty\let\ufop@templang\ufop@langgroup\fi
    \expandafter\ufop@parsedate#2---\relax
  \endgroup
}
\def\ufop@parsedate#1-#2-#3-#4\relax{%
  \@ufopdateerrortrue
  \def\ufop@year{\number#1}\def\ufop@month{\number#2}%
  \def\ufop@day{#3}\ifx\ufop@day\empty\else\def\ufop@day{\number#3}\fi
  \ifx\ufop@month\empty\else\ifnum#1>99\relax
    \@ufopdateerrorfalse
    \csname ufop@expanddate@\ufop@templang\endcsname
  \fi\fi
  \if@ufopdateerror
    \ClassError{ufopthesis}{Invalid date format.^^J%
    When using the `date=' option for \string\ufopthesis\space or the
    \string\date\string{...\string} command,^^J%
    always provide the date in the format yyyy-mm-dd or yyyy-mm^^J%
    (note the year-month-day order and the hyphen separator)}%
    \endinput
  \fi
}
%    \end{macrocode}

% \DescribeMacro{\ufop@lang@brazil}%
% \DescribeMacro{\ufop@lang@english}%
% Now the localization tables follow. You can add new translations here,
% provided that \emph{all} entries are translated!
%
%    \begin{macrocode}
\def\ufop@expandmonth@brazil#1{%
  \ifcase\ufop@month\or
  janeiro\or fevereiro\or mar\c{c}o\or abril\or maio\or junho\or
  julho\or agosto\or setembro\or outubro\or novembro\or dezembro\fi
}
\def\ufop@expanddate@brazil{%
  \edef\text@month{\ufop@expandmonth@brazil{\ufop@month}}%
  \ifx\ufop@day\empty\expandafter\MakeUppercase\text@month
    \else \two@digits\ufop@day\ de\ \text@month\fi
  \ de\ \ufop@year
}
\def\ufop@lang@brazil{%
  \@extraporttitlepagefalse

  \def\@deflang@title{%
    \@ifundefined{@portuguesetitle}{\@title}{\@portuguesetitle}}
  \def\@deflang@brokentitle{%
    \@ifundefined{@portuguesebrokentitle}{\@brokentitle}%
    {\@portuguesebrokentitle}}
  \def\@deflang@university{%
    \@ifundefined{@portugueseuniversity}{\@university}%
    {\@portugueseuniversity}}
  \def\@deflang@course{%
    \@ifundefined{@portuguesecourse}{\@course}{\@portuguesecourse}}
  \def\@deflang@course{%
    \@ifundefined{@portugueseshortcourse}{\@shortcourse}{\@portugueseshortcourse}}



  \def\ufop@titleuppertext{%
    \@deflang@university\\%
    Instituto de Ci�ncias Exatas\\
    
    \if@phd Programa de P�s-Gradua��o em \fi
    \if@msd P�s-Gradua��o em \fi
    \@deflang@course}

  \def\ufop@mono{Bacharel}
  \def\ufop@msc{Mestre}
  \def\ufop@phd{Doutor}
  \def\ufop@monodoc{Monografia}
  \def\ufop@monodocprop{Proposta de monografia}
  \def\ufop@mscdoc{Disserta\c{c}\~{a}o}
  \def\ufop@mscdocprop{Proposta de disserta\c{c}\~{a}o}
  \def\ufop@phddoc{Tese}
  \def\ufop@phddocproj{Projeto de tese}
  \def\ufop@approvaltitle{Folha de Aprova\c{c}\~{a}o}
  \def\ufop@advisor{Orientador}
  \def\ufop@coadvisor{Co-orientador}
  \def\ufop@docdescription{%
    \@documenttype{} apresentad\ufop@doctype@gender{} ao Curso de
    \if@phd % this is a PhD thesis
    P\'{o}s-Gradua\c{c}\~{a}o em 
    \fi
    \if@msd % this is a PhD thesis
    P\'{o}s-Gradua\c{c}\~{a}o em 
    \fi
    \@deflang@course{} da
    \@deflang@university{} como requisito parcial para a obten\c{c}\~{a}o
    do grau de \@degree{} em \@deflang@shortcourse.}
  \def\ufop@approvaltext{%
    \@documenttype{} defendid\ufop@doctype@gender{} e
    aprovad\ufop@doctype@gender{} pela banca examinadora constitu\'{i}da
    por:}
  \def\ufop@abstract{Resumo\if@phd\if@proposal\else\ Estendido\fi\fi}
  \def\ufop@acknowledgments{Agradecimentos}

  \def\ufop@set@phddocproj{\def\ufop@doctype@gender{o}}
  \def\ufop@set@phddoc{\def\ufop@doctype@gender{a}}
  \def\ufop@set@mscdocprop{\def\ufop@doctype@gender{a}}
  \def\ufop@set@mscdoc{\def\ufop@doctype@gender{a}}

  \def\ufopbibauthorsep{;}
  \def\ufopbibauthorlastsep{}
  \def\ufopbibauthorand{e}
  \def\ufopciteauthorand{e}
  \def\ufopbibpagestring{p.\@}
  \def\ufopbibpagesstring{pp.\@}
  \def\ufopbibeditorstring{editor}
  \def\ufopbibeditorsstring{editores}
  \def\ufopbibeditionstring{edi��o}
}

\def\ufop@expandmonth@english#1{%
  \ifcase\ufop@month\or
  january\or february\or march\or april\or may\or june\or
  july\or august\or september\or october\or november\or december\fi
}
\def\ufop@expanddate@english{%
  \edef\text@month{\ufop@expandmonth@english{\ufop@month}}%
  \expandafter\MakeUppercase\text@month
  \ifx\ufop@day\empty\else\ \ufop@day,\fi
  \ \ufop@year
}
\def\ufop@lang@english{%
  \@extraporttitlepagetrue

  \def\@deflang@title{\@title}
  \def\@deflang@brokentitle{\@brokentitle}
  \def\@deflang@university{\@university}
  \def\@deflang@course{\@course}
  \def\@deflang@shortcourse{\@shortcourse}

  \def\ufop@titleuppertext{%
    \@deflang@university\\%
    Instituto de Ci�ncias Exatas\\
    \if@phd Graduate Program in \fi
    \if@msd Graduate Program in \fi
    \@deflang@course}

  \def\ufop@mono{Bachelor}
  \def\ufop@msc{Master}
  \def\ufop@phd{Doctor}
  \def\ufop@mscdoc{Monograph}
  \def\ufop@mscdocprop{Monograph proposal}
  \def\ufop@mscdoc{Dissertation}
  \def\ufop@mscdocprop{Dissertation proposal}
  \def\ufop@phddoc{Thesis}
  \def\ufop@phddocproj{Thesis project}
  \def\ufop@approvaltitle{Approval Sheet}
  \def\ufop@advisor{Advisor}
  \def\ufop@coadvisor{Co-advisor}
  \def\ufop@docdescription{%
    \@documenttype{} presented to the Graduate Program in \@course{} of the
    \@university{} in partial fulfillment of the requirements for the
    degree of \@degree{} in \@course.}
  \def\ufop@approvaltext{%
    \@documenttype{} presented to and approved by the committee
    composed by:}
  \def\ufop@abstract{Abstract}
  \def\ufop@acknowledgments{Acknowledgments}

  \let\ufop@set@phddocproj\relax
  \let\ufop@set@phddoc\relax
  \let\ufop@set@mscdocprop\relax
  \let\ufop@set@mscdoc\relax

  \def\ufopbibauthorsep{,}
  \def\ufopbibauthorlastsep{,}
  \def\ufopbibauthorand{and}
  \def\ufopciteauthorand{and}
  \def\ufopbibpagestring{page}
  \def\ufopbibpagesstring{pages}
  \def\ufopbibeditorstring{editor}
  \def\ufopbibeditorsstring{editors}
  \def\ufopbibeditionstring{edition}
}
%    \end{macrocode}
% \subsection{Title pages}
%
% \DescribeMacro{\makecoverpage}%
% \DescribeMacro{\maketitlepage}%
% Generally a thesis contains two title pages: the first one, sometimes called
% \emph{cover page} or \emph{half title page}, contains only the institution
% name, author name, thesis title, city and date. The second one, the
% \emph{title page}, is quite similar, except that it contains a short
% descriptive text about the document.
%
% If the document isn't written in Portuguese, the |\maketitlepage| actually
% makes two title pages: one in Portuguese and other in the document language.
%
%    \begin{macrocode}
\newcommand{\makecoverpage}{%
  \if@showcover\begingroup
    \ufop@deflanguage{brazil}\ufop@selectlanguage
    \ufop@maketitlepage{\z@}
  \endgroup\fi
}
\newcommand{\maketitlepage}{%
  \if@showtitle
    \if@extraporttitlepage\begingroup
      \ufop@deflanguage{brazil}\ufop@selectlanguage
      \ufop@maketitlepage{\@ne}%
    \endgroup\fi
    \ufop@maketitlepage{\@ne}%
  \fi
}
%    \end{macrocode}

% \DescribeMacro{\ufop@docdescriptionwidth}%
% \DescribeMacro{\ufop@makedocdescription}%
% These macros are intended to build the small text that describes the
% document, which is part of the title page.
%
%    \begin{macrocode}
\def\ufop@docdescriptionwidth{0.5\textwidth}

\def\ufop@makedocdescription{%
  \begin{minipage}{\ufop@docdescriptionwidth}
    \ufop@docdescription
  \end{minipage}
}
%    \end{macrocode}

% \DescribeMacro{\ufop@titpag@fmtupper}%
% \DescribeMacro{\ufop@titpag@fmttitle}%
% \DescribeMacro{\ufop@titpag@fmtauthor}%
% \DescribeMacro{\ufop@titpag@fmtaddress}%
% \DescribeMacro{\ufop@titpag@fmtdate}%
% These macros are provided to format the corresponding informations about each
% cover/title page segment. If you want to change the formatting, please alter
% these macros, \emph{not} |\ufop@maketitlepage|!
%
%    \begin{macrocode}
\def\ufop@titpag@fmtupper#1{\large\textsc{#1}}
\def\ufop@titpag@fmttitle#1{\Large\textbf{\BreakableUppercase{#1}}}
\def\ufop@titpag@fmtauthor#1{\large\BreakableUppercase{#1}}
\def\ufop@titpag@fmtaddress#1{\large #1}
\def\ufop@titpag@fmtdate#1{\large #1}
%    \end{macrocode}

% \DescribeMacro{\ufop@maketitlepage}%
% This macro builds the cover and the title page. It's called with a mandatory
% argument, which is zero (|\z@|) to build the cover page or one (|\@ne|) to
% build the title page.
%
%    \begin{macrocode}
\def\ufop@maketitlepage#1{%
  \begin{titlingpage}
  \begin{singlespace}
  {\centering
    \vbox to48ex{%
      \ifx#1\z@
        {\ufop@titpag@fmtauthor{\@author}}%
      \else
        {\ufop@titpag@fmtupper{\ufop@titleuppertext}}%
      \fi
      \par\vfil
    }%

    \begin{doublespace}
    {\ufop@titpag@fmttitle{\@deflang@brokentitle}}\par

    \vspace{10ex}
    \end{doublespace}

    \ifx#1\@ne
      \hspace*{\stretch{1}}\ufop@makedocdescription\par
    \fi

    \vspace{\stretch{1}}

    \ifx#1\@ne
      {\ufop@titpag@fmtauthor{\@author}}\par
      \vspace{30ex}
    \fi

    {\ufop@titpag@fmtaddress{\@address}}\par
    \vspace{1ex}

    {\ufop@titpag@fmtdate{\ufop@formatdate{\@date}}}\par
  }
  \end{singlespace}
  \end{titlingpage}
}
%    \end{macrocode}


% \subsection{The Comitee and the Approval Sheet}
%
% There are some macros to handle the comitee list. These were built to accept
% any number of members in the comitee.
%
% First an internal count to store the comitee size:
%
%    \begin{macrocode}
\newcounter{ufop@banca}
%    \end{macrocode}

% \DescribeMacro{\addtocomitee}%
% Now the user macro that adds a member to the comitee: |\addtocomitee|
% \oarg{relationship} \marg{name} \marg{title} \marg{institution}. The
% \oarg{relationship} optional argument specifies the relationship between this
% member and the student, generally ``Advisor'' or ``Co-advisor''. Note that
% there are two macros, |\advisor| and |\coadvisor|, that adds the appropriate
% relationship word according to the current translation (see below).
%
%    \begin{macrocode}
\newcommand{\addtocomitee}[4][\relax]{%
  \addtocounter{ufop@banca}{1}
  \expandafter\def\csname ufop@banca@\theufop@banca name\endcsname
    {#2}
  \expandafter\def\csname ufop@banca@\theufop@banca title\endcsname
    {#3}
  \def\ufop@temp{#4}
  \if\ufop@temp.
    \expandafter\def\csname ufop@banca@\theufop@banca instit\endcsname
      {\@university}
  \else
    \expandafter\def\csname ufop@banca@\theufop@banca instit\endcsname
      {#4}
  \fi
  \ifx#1\relax\else
    \expandafter\def\csname ufop@banca@\theufop@banca rel\endcsname
      {#1}
  \fi
}
%    \end{macrocode}

% \DescribeMacro{\advisor}%
% \DescribeMacro{\coadvisor}%
% As mentioned before, these two macros just call |\addtocomitee| with the
% correct relationship word. Use: |\advisor| \marg{name} \marg{title}
% \marg{institution} (or |\coadvisor|).
%
%    \begin{macrocode}
\newcommand{\advisor}[3]{\addtocomitee[\ufop@advisor]{#1}{#2}{#3}}
\newcommand{\coadvisor}[3]{\addtocomitee[\ufop@coadvisor]{#1}{#2}{#3}}
%    \end{macrocode}

% \DescribeMacro{\ufop@fmtbancatitle}%
% \DescribeMacro{\ufop@fmtbancaname}%
% \DescribeMacro{\ufop@fmtbancainstit}%
% \DescribeMacro{\ufop@fmtbancarel}%
% These macros are provided to format the corresponding informations about each
% comitee member. If you want to change the comitee formatting, please alter
% these macros, \emph{not} |\ufop@makecomiteelist|!
%
%    \begin{macrocode}
\def\ufop@fmtbancatitle{}
\def\ufop@fmtbancaname{\textsc}
\def\ufop@fmtbancainstit{}
\def\ufop@fmtbancarel{}
%    \end{macrocode}

% \DescribeMacro{\ufop@makecomiteelist}%
% This is the macro that builds the comitee list, which is part of the approval
% sheet.
%
%    \begin{macrocode}
\newcommand{\ufop@makecomiteelist}{%
  \setcounter{ufop@banca}{1}
  \loop\expandafter\ifx\csname ufop@banca@\theufop@banca name\endcsname\relax
    \else
    {\centering
    \vspace{\stretch{1}}
    {\ufop@fmtbancatitle{%
      \csname ufop@banca@\theufop@banca title\endcsname}}~%
    {\ufop@fmtbancaname{%
      \csname ufop@banca@\theufop@banca name\endcsname}}%
    \@ifundefined{ufop@banca@\theufop@banca rel}{}{%
      \ -- {\ufop@fmtbancarel{%
        \csname ufop@banca@\theufop@banca rel\endcsname}}%
    }%
    \\%
    {\ufop@fmtbancainstit{%
      \csname ufop@banca@\theufop@banca instit\endcsname}}\\%
    }
    \stepcounter{ufop@banca}
  \repeat
  \vspace{\stretch{1.5}}
}
%    \end{macrocode}

% \DescribeMacro{\ufop@apprpag@fmtuniv}%
% \DescribeMacro{\ufop@apprpag@fmttitle}%
% \DescribeMacro{\ufop@apprpag@fmtauthor}%
% \DescribeMacro{\ufop@apprpag@fmttext}%
% \DescribeMacro{\ufop@apprpag@fmtaddrdate}%
% \DescribeMacro{\ufop@makecomiteelist}%
% These macros are provided to format the corresponding informations. If you
% want to change the comitee formatting, please alter these macros, \emph{not}
% |\makeapproval|!
%
%    \begin{macrocode}
\def\ufop@apprpag@fmtuniv#1{\normalsize\MakeUppercase{#1}}
\def\ufop@apprpag@fmttitle#1{\large\MakeUppercase{#1}}
\def\ufop@apprpag@fmtauthor#1{\large\BreakableUppercase{#1}}
\def\ufop@apprpag@fmttext{}
\def\ufop@apprpag@fmtaddrdate{}
%    \end{macrocode}

% \DescribeMacro{\makeapproval}%
% This macro typesets the approval sheet.
%
%    \begin{macrocode}
\newcommand{\makeapproval}{%
  \if@showapproval
    \begingroup
    \ufop@deflanguage{brazil}\ufop@selectlanguage
    \begin{titlingpage}
    \begin{singlespace}
    {\centering
      \ifx\@logo\@undefined
        {\ufop@apprpag@fmtuniv{\@deflang@university}}\par
        \vspace{12ex}
      \else
        \raisebox{-0.5\height}{\includegraphics{\@logo}}\quad
        {\ufop@apprpag@fmtuniv{\@deflang@university}}\par
        \vspace{6ex}
      \fi

      {\ufop@apprpag@fmttitle{\ufop@approvaltitle}}\par
      \vspace{5ex}

      {\Large\@deflang@brokentitle\par}
      \vspace{5ex}

      {\ufop@apprpag@fmtauthor{\@author}}\par
      \vspace{5ex}
    }

    {\ufop@apprpag@fmttext{\ufop@approvaltext}}\par
    \vspace{4ex}

    \ufop@makecomiteelist

    {\centering
      {\ufop@apprpag@fmtaddrdate{\@address, \ufop@formatdate{\@date}}}%
      \par
    }
    \end{singlespace}
    \end{titlingpage}
    \endgroup
  \fi
}
%    \end{macrocode}

% \subsection{Remaining pages}
%
% The other front matter pages are all based on external |.tex| files, since
% there's no way to automatically typeset them :-) . All macros expect the name
% of a |.tex| file as an argument (it can include relative or absolute paths),
% but beware: \TeX{} doesn't like paths or filenames containing spaces or
% special characters, such as accented letters. Please do not stick to
% filenames like ``\texttt{Cap\'{i}tulo 1.tex}''. Also, use slashes (|/|)
% instead of backslashes (|\|) when specifying paths, even under MS Windows.
%
% None of the files need a |\section| command; it's issued automatically by
% these macros.
%
% \DescribeMacro{\includeabstract}%
% This macro includes the abstract page. The use is: |\includeabstract|
% \oarg{title} \marg{filename}. \meta{title} is the abstract title, which
% defaults to ``Abstract'' translated to the current language. (It may be used
% to include an alternative abstract written in a foreign language.)
%
%    \begin{macrocode}
\newcommand{\includeabstract}[2][]{%
  \if@showabstract
%    \begin{titlingpage}
    \ufop@redefchaptitlefont
    \def\@temp{#1}
    \ifx\@temp\empty\def\@temp{\ufop@abstract}\else\def\@temp{#1}\fi
    \chapter*{\@temp}
    \input{#2}
%    \end{titlingpage}
    \cleardoublepage
  \fi
}
%    \end{macrocode}

% \DescribeMacro{\includededication}%
% This macro includes the dedication page. The use is: |\includededication|
% \marg{filename}.
%
%    \begin{macrocode}
\newcommand{\includededication}[1]{%
  \if@showdedication
%    \begin{titlingpage}
    \vspace*{\stretch{1}}
    \textit{\input{#1}}
    \vspace*{\stretch{1}}
%    \end{titlingpage}
    \cleardoublepage
  \fi
}
%    \end{macrocode}

% \DescribeMacro{\includeack}%
% This macro includes the acknowledgment page. The use is: |\includeack|
% \marg{filename}.
%
%    \begin{macrocode}
\newcommand{\includeack}[1]{%
  \if@showack
%    \begin{titlingpage}
    \ufop@redefchaptitlefont
    \chapter*{\ufop@acknowledgments}
    \input{#1}
%    \end{titlingpage}
    \cleardoublepage
  \fi
}
%    \end{macrocode}


% \subsection{Page headings}
%
% The following code does all the necessary stuff to format the page headings.
%
%    \begin{macrocode}
\if@twoside
  \makeheadrule{headings}{\textwidth}{\normalrulethickness}
  \makeoddhead{headings}{\textsc{\rightmark}}{}{\thepage}
  \makeevenhead{headings}{\thepage}{}{\textsc{\leftmark}}

  \copypagestyle{contents}{plain}

  \copypagestyle{listoffigures}{plain}

  \copypagestyle{listoftables}{plain}

  \copypagestyle{bibliography}{headings}
  \makeoddhead{bibliography}{\textsc{\bibname}}{}{\thepage}
  \makeevenhead{bibliography}{\thepage}{}{\textsc{\bibname}}
\else
  \makeheadrule{headings}{\textwidth}{\normalrulethickness}
  \makeoddhead{headings}{\textsc{\leftmark}}{}{\thepage}

  \copypagestyle{contents}{plain}

  \copypagestyle{listoffigures}{plain}

  \copypagestyle{listoftables}{plain}

  \copypagestyle{bibliography}{headings}
  \makeoddhead{bibliography}{\textsc{\bibname}}{}{\thepage}
\fi

\def\ufop@chaptermark#1{%
  \markboth{%
    \ifnum\c@secnumdepth>\m@ne
      \if@mainmatter
        \if@twoside\@chapapp\ \fi
        \thechapter.
      \fi
    \fi
    #1}{}}%

\def\ufop@sectionmark#1{%
  \markright{%
    \ifnum\c@secnumdepth>\z@\thesection. \ \fi
    #1}}%

\def\ufoprestoremarks{%
  \let\chaptermark=\ufop@chaptermark
  \let\sectionmark=\ufop@sectionmark
}

\ufoprestoremarks
\g@addto@macro\mainmatter{\pagestyle{headings}\ufoprestoremarks}
%    \end{macrocode}


% \subsection{Table of contents, list of figures and tables}
%
% \DescribeMacro{\tableofcontents}%
% \DescribeMacro{\listoffigures}%
% \DescribeMacro{\listoftables}%
% The commands that builds the table of contents and the list of figures and
% tables are hacked to include the correct formatting of page headers.
%
%    \begin{macrocode}
\let\ufop@orig@tableofcontents=\tableofcontents
\renewcommand{\tableofcontents}{{%
  \if@showtoc\begingroup
    \ufop@redefchaptitlefont
    \let\ufop@save@tableofcontents=\tableofcontents
    \let\tableofcontents=\ufop@orig@tableofcontents
    \pagestyle{contents}
    \tableofcontents*
    \cleardoublepage
  \endgroup\fi
}}

\let\ufop@orig@listoffigures=\listoffigures
\renewcommand{\listoffigures}{{%
  \if@showlof\begingroup
    \ufop@redefchaptitlefont
    \let\ufop@save@listoffigures=\listoffigures
    \let\listoffigures=\ufop@orig@listoffigures
    \pagestyle{listoffigures}
    \listoffigures*
    \cleardoublepage
  \endgroup\fi
}}

\let\ufop@orig@listoftables=\listoftables
\renewcommand{\listoftables}{{%
  \if@showlot\begingroup
    \ufop@redefchaptitlefont
    \let\ufop@save@listoftables=\listoftables
    \let\listoftables=\ufop@orig@listoftables
    \pagestyle{listoftables}
    \listoftables*
    \cleardoublepage
  \endgroup\fi
}}
%    \end{macrocode}


% \subsection{The ``let-me-do-everything'' macro}
%
% \DescribeMacro{\ufopthesis}%
% This is the ``magic macro'' that typesets all the introductory pages. It
% receives an argument with a series of key/value pairs, which are described in
% Section \ref{sec:ufopthesis}. Please read it!
%
%    \begin{macrocode}

\def\ufop@setportabstractfile#1{\def\ufop@portabstractfile{#1}}
\def\ufop@setabstractfile#1{\def\ufop@abstractfile{#1}}
\def\ufop@setenabstractfile#1{\def\ufop@enabstractfile{#1}}
\def\ufop@setdedicationfile#1{\def\ufop@dedicationfile{#1}}
\def\ufop@setackfile#1{\def\ufop@ackfile{#1}}

\define@key{ufop}{title}{\title{#1}}
\define@key{ufop}{brokentitle}{\brokentitle{#1}}
\define@key{ufop}{author}{\author{#1}}
\define@key{ufop}{university}{\university{#1}}
\define@key{ufop}{course}{\course{#1}}
\define@key{ufop}{shortcourse}{\shortcourse{#1}}
\define@key{ufop}{portuguesetitle}{\portuguesetitle{#1}}
\define@key{ufop}{portuguesebrokentitle}{\portuguesebrokentitle{#1}}
\define@key{ufop}{portugueseuniversity}{\portugueseuniversity{#1}}
\define@key{ufop}{portuguesecourse}{\portuguesecourse{#1}}
\define@key{ufop}{portugueseshortcourse}{\portugueseshortcourse{#1}}
\define@key{ufop}{address}{\address{#1}}
\define@key{ufop}{date}{\date{#1}}
\define@key{ufop}{logo}{\logo{#1}}
\define@key{ufop}{advisor}{\advisor#1}
\define@key{ufop}{coadvisor}{\coadvisor#1}
\define@key{ufop}{member}{\addtocomitee#1}
\define@key{ufop}{portugueseabstract}{\ufop@setportabstractfile{#1}}
\define@key{ufop}{abstract}{\ufop@setabstractfile{#1}}
\define@key{ufop}{englishabstract}{\ufop@setenabstractfile{#1}}
\define@key{ufop}{dedication}{\ufop@setdedicationfile{#1}}
\define@key{ufop}{ack}{\ufop@setackfile{#1}}

\newcommand{\ufopthesis}[1]{%
  \setkeys{ufop}{#1}

  \frontmatter
  \pagestyle{plain}

  \makecoverpage
  \maketitlepage
  \makeapproval
  \@ifundefined{ufop@portabstractfile}{}{%
    \begingroup
      \ufop@deflanguage{brazil}\ufop@selectlanguage
      \includeabstract{\ufop@portabstractfile}%
    \endgroup}
  \@ifundefined{ufop@abstractfile}{}
    {\includeabstract{\ufop@abstractfile}}
  \@ifundefined{ufop@enabstractfile}{}{%
    \begingroup
      \ufop@deflanguage{english}\ufop@selectlanguage
      \includeabstract{\ufop@enabstractfile}%
    \endgroup}
  \@ifundefined{ufop@dedicationfile}{}
    {\includededication{\ufop@dedicationfile}}
  \@ifundefined{ufop@ackfile}{}
    {\includeack{\ufop@ackfile}}

  \tableofcontents
  \listoffigures
  \listoftables

  \if@putmainmatter\mainmatter\fi
}
%    \end{macrocode}

%
% \subsection{Bibliography stuff}
%
% Here are all bibliography-related macros, options, processing, etc..
%
% Let's start with macros that store some customizable options:
%
%    \begin{macrocode}
\def\ufop@bibauthorformat{}
\def\ufop@bibtitleformat{}
\def\ufop@bibbtitleformat{\itshape}
\def\ufop@bibbooktitleformat{\itshape}
\def\ufop@bibjournalformat{\itshape}

\def\ufop@norepstring{---------}
\def\ufopbibinstring{In}
%    \end{macrocode}

% The following macros are meant to be automatically used only inside the
% |.bbl| file (which is compiled by Bib\TeX):
%
%    \begin{macrocode}
\newcommand{\ufopbibauthordoformat}[1]{%
  \if@bibnorepauthor
    \def\ufop@currentauthor{#1}%
    \ifx\ufop@lastauthor\ufop@currentauthor\ufop@norepstring
      \else{\ufop@bibauthorformat#1}\fi
    \def\ufop@lastauthor{#1}%
  \else
    {\ufop@bibauthorformat#1}%
  \fi
}
\newcommand{\ufopbibtitledoformat}[1]{{\ufop@bibtitleformat#1}}
\newcommand{\ufopbibbtitledoformat}[1]{{\ufop@bibbtitleformat#1}}
\newcommand{\ufopbibbooktitledoformat}[1]{{\ufop@bibbooktitleformat#1}}
\newcommand{\ufopbibjournaldoformat}[1]{{\ufop@bibjournalformat#1}}
%    \end{macrocode}

% Here comes the declaration of the |\ufopbibliography| options:
%
%    \begin{macrocode}
\define@key{ufopbib}{noauthorrepeat}[true]{\@bibnorepauthortrue}
\define@key{ufopbib}{noauthorrepstring}{\def\ufop@norepstring{#1}}
\define@key{ufopbib}{nobreakitems}[true]{\@nobreakitemstrue}
\define@key{ufopbib}{bibauthorand}{\def\ufopbibauthorand{#1}}
\define@key{ufopbib}{citeauthorand}{\def\ufopciteauthorand{#1}}
\define@key{ufopbib}{authorformat}{\def\ufop@bibauthorformat{#1}}
\define@key{ufopbib}{titleformat}{\def\ufop@bibtitleformat{#1}}
\define@key{ufopbib}{btitleformat}{\def\ufop@bibbtitleformat{#1}}
\define@key{ufopbib}{booktitleformat}{\def\ufop@bibbooktitleformat{#1}}
\define@key{ufopbib}{journalformat}{\def\ufop@bibjournalformat{#1}}
%    \end{macrocode}

% \DescribeMacro{\ufopbibliography}%
% Finally, here comes the magic macro. The mandatory argument is the name of
% the bibliography database file (the |.bib| file), without the extension.
%
%    \begin{macrocode}
\newcommand{\ufopbibliography}[2][]{%
  \setkeys{ufopbib}{#1}

  \backmatter

  \pagestyle{bibliography}
  \bibliographystyle{ufop}
  \begingroup
    \ufop@redefchaptitlefont
    \if@nobreakitems\raggedbottom\interlinepenalty=10000\relax\fi
    \bibliography{#2}
  \endgroup
}
%    \end{macrocode}

% \subsection{Post-processing hook}
%
% \DescribeMacro{\AtBeginDocument}%
% Finally, the |\AtBeginDocument| hook: it calls the macros to configure the
% language and the default line spacing. The |graphicx| package is also loaded
% if it was not done by the user, but the best is to manually include it ---
% since it's important to inform the target driver: |dvips|, |dvipdfm|, etc..
%
%    \begin{macrocode}
\AtBeginDocument{%
  \ufop@selectlanguage\ufop@defspacing
  \@ifpackageloaded{graphicx}{}{\RequirePackage{graphicx}}%
  \@ifpackageloaded{hyperref}{\RequirePackage{memhfixc}}{}%
}
%    \end{macrocode}
%
% \Finale
