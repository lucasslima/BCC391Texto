\chapter{Introdução}
\label{chapter:intro}

%Expor o tema\footnote{Um tema expressa a idéia central da pesquisa, aquilo que irá identificar o objeto de estudo da pesquisa ou a intenção do autor~\cite{Booth03,Denscombe2012}.} geral da monografia e fornecer as motivações contextuais que levaram o autor a conduzir o trabalho (e.g., dissertar sobre a importância e relevância, científica e/ou pra sociedade, do estudo/pesquisa que se pretende fazer de modo geral e/ou para a localidade/região)~\cite{Booth03,katz2009,Denscombe2012}. Além disso, nesta seção, podem ser apresentados, de forma sucinta, a definição de conceitos essenciais para o entendimento do trabalho. 

Nos últimos 10 anos, Software Livre (Free/Libre Open-Source    Software FLOSS) tem se tornado um dos grandes objetos e pesquisa de desenvolvimento da comunidade acadêmica. Existem hoje milhares de projetos FLOSS, em diversas áreas de aplicação. Centenas de \textit{software} livres são largamente utilizados, incluindo projetos que integram a infraestrutura da internet (e.g. sendmail, bind, Apache Hadoop), aplicações voltadas à usuários(e.g., Mozilla Firefox, LibreOffice), compiladores e interpretadores de linguagens de programação(e.g. Perl, Python, gcc), ambientes de desenvolvimento integrado (e.g., Eclipse) ou até mesmo \textit{software} empresarial (e.g., eGroupware, openCRX).


Muitos projetos FLOSS são desenvolvidos por times organizacionalmente e geograficamente distribuídos, no que é descrito como desenvolvimento comunitário\cite{lee2003firm}. Em alguns projetos, uma determinada organização toma a liderança na coordenação dos esforços da comunidade \cite{fitzgerald2006transformation}, como a fundação APACHE, foco de estudo deste trabalho. Nestes cenários, plataformas de hospedagem de código tais como \textit{Github} e \textit{SourceForge} são frequentemente utilizadas para apoiar a modelagem e implementação de projetos FLOSS. Embora, empresas tem aumentado consideravelmente sua participação em projetos FLOSS assim como contratando  desenvolvedores para trabalhar exclusivamente em tais projetos\cite{lakhaniwolf}.


Outra característica importante de FLOSS desenvolvidos pela comunidade caracteriza-se pelo fato dos desenvolvedores contribuírem como voluntários, sem remuneração; outros são remunerados por seus empregadores para trabalhar no projeto, mesmo que o projeto não contribua diretamente com a sua remuneração. Adicionalmente, o risco de \textit{"forking"} (criação de um projeto independente baseado na mesma base de código), apesar de incomum e desencorajado, limita a capacidade de seus líderes de disciplinar os membros colaboradores. Tais características fazem equipes de desenvolvimento FLOSS exemplos extremos de times auto-organizáveis distribuídos, consistentes com as condições encontradas por várias empresas quando recrutando e motivando profissionais ou desenvolvendo times distribuídos.
Como resultado, FLOSS se tornou uma parte integral da infraestrutura da sociedade moderna, tornando fundamental uma compreensão mais profunda de como se desenvolve.

%\section{Definição do Problema}



\section{Definição do Problema e Justificativas}

A compreensão do movimento FLOSS tornou-se importante pela sua própria dimensão\cite{crowston2012free}. FLOSS é hoje considerado um grande movimento social, envolvendo um estimado de 800,000 programadores ao redor do mundo assim como um fenômeno comercial envolvendo um grande número de empresas de desenvolvimento de \textit{software} de várias tamanhos, desde \textit{startups} até empresas bem estabelecidas. Do outro lado, milhões  de usuários tornaram-se dependentes de sistemas FLOSS como Linux, além de que a \textit{Internet} em si é altamente dependente de ferramentas desenvolvidas pelo movimento. Estudos recentes estimam que 87\% dos negócios situados nos Estados Unidos utilizam FLOSS. Estima-se que o custo para recriar a quantidade de código produzido pelo movimento FLOSS em 12 bilhões de euros. Tal base de código vem dobrando a cada 18 à 24 meses nos últimos 8 anos, e há projeções para indicar que continuará a crescer por vários anos\cite{ghosh2007economic}. 

Apesar de vários estudos terem abordado o movimento FLOSS, revisões bibliográficas da literatura previamente realizadas\cite{crowston2012free} mostram que 
há um grande número de trabalhos publicados visando a análise organizacional do projeto ou grupo em questão, com apenas 7\% dos trabalhos focados nos artefatos produzidos pelos mesmos. 

Muitos FLOSS caem na classificação de Lehman\cite{belady1976model} como softwares de tipo E(\textit{E-type Systems}). Estes sistemas têm como propósito modelar problemas do mundo real e através de seu uso, tornam-se parte do mundo que tentam modelar. Com o tempo, tais sistemas são modificados de acordo com novos requerimentos e mudanças desejadas por usuários, evoluindo de maneiras que não podem ser previstas ou especificadas previamente. Eles passam pelo chamado \textit{desenvolvimento perpétuo}\cite{lehman1996laws}, com novas versões de produção lançadas frequentemente. Mesmo que o desenvolvimento possa um dia chegar ao fim quando o uso do sistema é descontinuado, a mentalidade dos desenvolvedores é que o projeto progredirá indefinidamente.

Os estudos de Lehman sobre a evolução dos sistemas de \textit{software} tipo-E levaram a formulação de um conjunto de "leis" da evolução de \textit{software}. Estas incluem observações como sistemas que estão em constante uso inevitavelmente continuam a crescer, para se adaptarem ao ambiente. É importante também ressaltar que tais leis são gerais e não dependem da forma como o \textit{software} estudado é desenvolvido. Tais leis foram validadas pelo estudo da evolução do tamanho de grandes sistemas comerciais como o IBM OS/360\cite{lehman1979understanding,lehman1985program}. 

Vários estudos já foram realizados tentando verificar a validade das Leis de Lehman as ao contexto de \textit{Software} Livre obtendo resultados variados. Em grande parte dos estudos, há evidencias de validade apenas parcial das Leis de Lehman de acordo com os resultados encontrados\cite{lehman1979understanding,lehman1985program,israeli2010linux,carver2004impact}. Devido as diferentes métricas e à observações que à validade das leis varia dentre projetos, os autores argumentam que mais estudos são necessários para que haja uma melhor compreensão da evolução do \textit{Software} Livre\cite{scacchi2003understanding,neamtiu2013towards,israeli2010linux}.

Assim, com o intuito de avançar o estado da arte no que tange a compreensão da aplicabilidade das lei de Lehman no âmbito de \textit{software} livre, este trabalho analisa a evolução da plataforma livre conhecida como Apache Hadoop.

O Hadoop é um \textit{framework} utilizado para armazenamento e processamento distribuído grandes massas de dados, utilizado por grande parte das maiores empresas de software americanas\footnote{http://www.prnewswire.com/news-releases/altiors-altrastar---hadoop-storage-accelerator-and-optimizer-now-certified-on-cdh4-clouderas-distribution-including-apache-hadoop-version-4-183906141.html. Acesado em 10/08/2017}. Devido à sua utilização e a grande diversidade de desenvolvedores espalhados por diversas empresas\footnote{Lista dos mantenedores e seus respectivos empregadores pode ser encontrada em https://hadoop.apache.org/who.html} tornam o Hadoop um bom candidato para a verificação das Leis de Lehman. %largamente utilizado por empresas com grandes  devido ao seu tamanho e influência em diversas empresas e serviços hoje disponíveis na internet\footnote{ A lista completa de empresas e serviços baseados no Hadoop pode ser encontrada em https://wiki.apache.org/hadoop/PoweredBy}.

\section{Objetivos}

Nesta seção, são apresentados o Objetivo Geral (Subseção \ref{ssec:num1}) e os Objetivos Específicos (Subseção \ref{ssec:num2})
\subsection{Objetivo Geral} \label{ssec:num1}
Este trabalho visa analisar a evolução do Apache Hadoop através de métricas de tamanho, complexidade e qualidade previamente utilizadas na literatura\cite{israeli2010linux,lehman1979understanding, mccabe1976complexity}, e determinar se os projetos obedecem às Leis de Lehman sobre a evolução de software.

\subsection{Objetivos Específicos} \label{ssec:num2}
\begin{itemize}
\item Realizar uma revisão das leis de Lehman para a evolução de software.
\item Descrever de forma geral a evolução do Apache Hadoop.
\item Realizar um estudo comparativo dos resultados encontrados com as conclusões de trabalhos anteriores.
\end{itemize}


\section{Organização do Documento}

O restante deste documento é organizado da seguinte forma:


No Capítulo \ref{chapter:fundamentos}, é apresentada a fundamentação teórica do trabalho, revisando
os principais conceitos relacionados a manutenibilidade de código, avaliação de débito técnico e evolução de software.

No Capítulo \ref{chapter:trabalhosRelacionados}, uma revisão bibliográfica da literatura é apresentada, descrevendo estudos previamente realizados, suas conclusões e 
contribuições.

No Capítulo \ref{chapter:desenvolvimento}, são descritas as métricas e medições realizadas nos projetos.