\chapter{Conclusões}
\label{chapter:conclusion}
Sistemas de \textit{software Open-Source} são parte integral do mundo moderno, atuando em áreas diversas como infraestrutura de rede, aplicações para o usuário final, linguagens de programação, entre outros. Apesar de muito utilizados, tais sistemas são pouco compreendidos e pouco se sabe de sua evolução.

Por tal motivo, foi conduzido um estudo empírico da evolução do framework de desenvolvimento distribuído \textit{Apache Hadoop}, para avaliar a validade das Leis de Lehman sobre evolução de \textit{software} no mesmo. Como resultados, encontramos que as leis de Mudança Contínua, Auto-Regulação e Crescimento contínuo aparentam válidas para o Hadoop, enquanto não foi possível validar as Complexidade Crescente, Conservação da Estabilidade Organizacional, Conservação da Familiaridade, Qualidade Decrescente, e Sistema de Realimentação, de forma similar a encontrada em trabalhos anteriores\cite{neamtiu2013towards,israeli2010linux,skoulis2014open}. 

Este estudo procurou mostrar que as Leis de Lehman podem não se aplicar bem a projetos de \textit{software Open-Source}, assim evidenciando que há grandes oportunidades em pesquisa em evolução de \textit{software} no domínio. 

Para uma melhor definição do comportamento das Leis de Lehman no domínio de sistemas de \textit{software Open-Source}, podemos em trabalhos futuros aplicar a metodologia utilizada a outros projetos dentro da \textit{Apache Foundation} ou mesmo fora dela, para e avaliar se os resultados são similares.