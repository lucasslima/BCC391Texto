\chapter{Conclusões}
\label{chapter:conclusion}

O Linux Kernel é considerado hoje o maior projeto de software público
da história. Desde seu lançamento em 1991, vem impulsionando o desenvolvimento
do ecossistema GNU/Linux como um todo, e tornou-se um dos softwares
mais importantes do cenário atual. Estudar o funcionamento permite
que os envolvidos no projeto avaliem a saúde do mesmo, e continuem
a evoluir o projeto. Mesmo com a conclusão de trabalhos anteriores
de que a qualidade do código que a manutenibilidade do código se degradava
a cada novo lançamento, o Linux kernel continuou a ser um projeto
de sucesso ao longo dos anos, superando tais limitações.

Este trabalho procurou avaliar o estado de manutenibilidade do projeto,
através de uma análise do seu código fonte. Suas praticas de desenvolvimento
foram revisadas, e com base no método SQALE o projeto foi avaliado
utilizando o software SonarQube. Pode-se observar que mesmo possuindo
um fluxo elevado de atividade a manutenibilidade do projeto está em
bom estado, mantendo a classificação A, a melhor classificação de
manutenibilidade dentro da escala estabelecida pelo SonarQube. Tal
resultado refuta as previsões de Schach\cite{schach2002maintainability}.
Os resultados também reforçam a avaliação realizada em \cite{israeli2010linux},
mostrando que o software continua evoluindo com uma boa classificação
de manutenibilidade.